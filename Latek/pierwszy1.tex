\documentclass[11pt, a4paper]{article}
\usepackage{polski}
\usepackage[utf8]{inputenc}
\usepackage{graphicx}
\begin{document}
Pierwszy dokument w systemie \LaTeX

\Large {Wielki Wódz} \small \textbf {Kim II Sung} \textit{13 kwietnia 1952 roku} \tiny w małej wiosce Mangyondawei 
\normalsize 

\$ \& \% \# \{{ \}

Dzisiaj jest \today 

To jest \hspace{3.5cm}odstep równy 3,5cm.

To \emph{słowo} jest wyróżnione Z kolei to \textsf{słowo} jest napisane czcionką bezszerfyfową.\texttt{To zdanie jest napisane czcionką groteskową.}

{\large Ten akapit jest trochę większy. Możemy nadal stosować inne czcionki, np. \textbf{pogrubiona} i będą one również powiększone.} 

{\small W tym akapicie niektóre słowa są {\tiny W tym akapicie niektóre słowa są mnijsze. Do tego mogą być napisane \textsc{kapitalikami} lub 
\textit{kursywą}.} \newline

To jest pierwsza linia. \newline
To jest druga linia. \newline
	W tym zdaniu wykorzystam twarda spacje,  która została użyta tylko w ~ tym miejscu.
	
17-go grudnia, godz. 18:00-23:00 odbędzie się Festiwal. \newline
Festiwal --- Uroczystość złożona z szeregu imprez artystycznych.
\newline

\begin{enumerate}
\item Baza danych:
\begin{itemize}
\item Relacyjne,
\item NoSQL.
\end{itemize}
\item Języki Programowania:
\begin{flushleft}
\item \textbf{C} - strukturalny język programowania
\item \textbf{Java} - obiektowy język programowania,
\item \textbf{Haskell} - funkcjonalny język programoania,
\item \textbf{Prolog} - logiczny język programowania.
\end{flushleft}

\end{enumerate}

\begin{flushleft}
do \\
lewej
\end{flushleft}
\begin{center}
do \\ 
środka
\end{center}
\begin{flushright}
do \\
prawej
\end{flushright}
\begin{tabular}{r|lc}
pierwszy tekst & drugi tekst & trzeci tekst \\
\hline 
pierwszy inny tekst & drugi inny tekst & trzeci inny tekst \\
  & drugi jeszcze inny tekst &
\end{tabular}

Przypis \footnote{Tutaj jest treść przypisana}

\includegraphic{grafika.jpeg}


\end{document}
	